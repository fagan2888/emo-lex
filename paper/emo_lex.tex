\documentclass[sigconf, authordraft]{acmart}

\usepackage{booktabs} % For formal tables


% Copyright
%\setcopyright{none}
%\setcopyright{acmcopyright}
%\setcopyright{acmlicensed}
\setcopyright{rightsretained}
%\setcopyright{usgov}
%\setcopyright{usgovmixed}
%\setcopyright{cagov}
%\setcopyright{cagovmixed}


% DOI
\acmDOI{10.1145/nnnnnnn.nnnnnnn}

% ISBN
\acmISBN{978-x-xxxx-xxxx-x/YY/MM}

%Conference
\acmConference[GECCO '18]{the Genetic and Evolutionary Computation
Conference 2018}{July 15--19, 2018}{Kyoto, Japan}
\acmYear{2018}
\copyrightyear{2018}

\acmPrice{15.00}

\acmSubmissionID{123-A12-B3}

\begin{document}
\title{An analysis of $\epsilon$-lexicase selection for large-scale many-objective optimization}
%\titlenote{Produces the permission block, and copyright information}
%\subtitle{Extended Abstract}
%\subtitlenote{The full version of the author's guide is available as
%  \texttt{acmart.pdf} document}


\author{William La Cava}
%\authornote{Dr.~Trovato insisted his name be first.}
%\orcid{1234-5678-9012}
\affiliation{%
  \institution{University of Pennsylvania}
  \streetaddress{3700 Hamilton Walk}
  \city{Philadelphia} 
  \state{PA} 
  \postcode{19143}
}
\email{lacava@upenn.edu}

\author{Jason H. Moore}
%\authornote{The secretary disavows any knowledge of this author's actions.}
\affiliation{%
  \institution{University of Pennsylvania}
  \streetaddress{3700 Hamilton Walk}
  \city{Philadelphia} 
  \state{PA} 
  \postcode{19143}
}
\email{moorejh@upenn.edu}

% The default list of authors is too long for headers.
\renewcommand{\shortauthors}{B. Trovato et al.}

\begin{abstract}
In this paper we adapt $\epsilon$-lexicase selection, a parent selection strategy designed for genetic programming, to solve many-objective optimization problems. $\epsilon$-lexicase selection has been shown to perform well in regression due to its use of full program semantics for conducting selection. A recent theoretical analysis showed that the behavior of this selection strategy is to preserve individuals located near the boundaries of the Pareto front in semantic space. We hypothesize that this strategy, of biasing search to extreme positions in objective space may be beneficial as the number of objectives in problems increases. Here, we replace program semantics with objective fitness to conduct selection for many-objective optimization. We derive the probabilities of selection under lexicase selection to illustrate its behavior. We then compare this method to multi-objective optimization methods from literature on problems ranging from 5 to 100 objectives. We find that $\epsilon$-lexicase selection outperforms state-of-the-art optimization algorithms in terms of convergence to the Pareto front, spread of solutions, and wallclock runtime across problems and objectives.   
\end{abstract}

%
% The code below should be generated by the tool at
% http://dl.acm.org/ccs.cfm
% Please copy and paste the code instead of the example below. 
%
%\begin{CCSXML}
%<ccs2012>
% <concept>
%  <concept_id>10010520.10010553.10010562</concept_id>
%  <concept_desc>Computer systems organization~Embedded systems</concept_desc>
%  <concept_significance>500</concept_significance>
% </concept>
% <concept>
%  <concept_id>10010520.10010575.10010755</concept_id>
%  <concept_desc>Computer systems organization~Redundancy</concept_desc>
%  <concept_significance>300</concept_significance>
% </concept>
% <concept>
%  <concept_id>10010520.10010553.10010554</concept_id>
%  <concept_desc>Computer systems organization~Robotics</concept_desc>
%  <concept_significance>100</concept_significance>
% </concept>
% <concept>
%  <concept_id>10003033.10003083.10003095</concept_id>
%  <concept_desc>Networks~Network reliability</concept_desc>
%  <concept_significance>100</concept_significance>
% </concept>
%</ccs2012>  
%\end{CCSXML}

%\ccsdesc[500]{Computer systems organization~Embedded systems}
%\ccsdesc[300]{Computer systems organization~Redundancy}
%\ccsdesc{Computer systems organization~Robotics}
%\ccsdesc[100]{Networks~Network reliability}


\keywords{ACM proceedings, \LaTeX, text tagging}


\maketitle

\input{samplebody-conf}

\bibliographystyle{ACM-Reference-Format}
\bibliography{sample-bibliography} 

\end{document}
